% Options for packages loaded elsewhere
\PassOptionsToPackage{unicode}{hyperref}
\PassOptionsToPackage{hyphens}{url}
%
\documentclass[
]{article}
\usepackage{amsmath,amssymb}
\usepackage{lmodern}
\usepackage{iftex}
\ifPDFTeX
  \usepackage[T1]{fontenc}
  \usepackage[utf8]{inputenc}
  \usepackage{textcomp} % provide euro and other symbols
\else % if luatex or xetex
  \usepackage{unicode-math}
  \defaultfontfeatures{Scale=MatchLowercase}
  \defaultfontfeatures[\rmfamily]{Ligatures=TeX,Scale=1}
\fi
% Use upquote if available, for straight quotes in verbatim environments
\IfFileExists{upquote.sty}{\usepackage{upquote}}{}
\IfFileExists{microtype.sty}{% use microtype if available
  \usepackage[]{microtype}
  \UseMicrotypeSet[protrusion]{basicmath} % disable protrusion for tt fonts
}{}
\makeatletter
\@ifundefined{KOMAClassName}{% if non-KOMA class
  \IfFileExists{parskip.sty}{%
    \usepackage{parskip}
  }{% else
    \setlength{\parindent}{0pt}
    \setlength{\parskip}{6pt plus 2pt minus 1pt}}
}{% if KOMA class
  \KOMAoptions{parskip=half}}
\makeatother
\usepackage{xcolor}
\IfFileExists{xurl.sty}{\usepackage{xurl}}{} % add URL line breaks if available
\IfFileExists{bookmark.sty}{\usepackage{bookmark}}{\usepackage{hyperref}}
\hypersetup{
  pdftitle={Intro to Urban Analytics},
  pdfauthor={Bon Woo Koo \& Subhrajit Guhathakurta},
  hidelinks,
  pdfcreator={LaTeX via pandoc}}
\urlstyle{same} % disable monospaced font for URLs
\usepackage[margin=1in]{geometry}
\usepackage{graphicx}
\makeatletter
\def\maxwidth{\ifdim\Gin@nat@width>\linewidth\linewidth\else\Gin@nat@width\fi}
\def\maxheight{\ifdim\Gin@nat@height>\textheight\textheight\else\Gin@nat@height\fi}
\makeatother
% Scale images if necessary, so that they will not overflow the page
% margins by default, and it is still possible to overwrite the defaults
% using explicit options in \includegraphics[width, height, ...]{}
\setkeys{Gin}{width=\maxwidth,height=\maxheight,keepaspectratio}
% Set default figure placement to htbp
\makeatletter
\def\fps@figure{htbp}
\makeatother
\setlength{\emergencystretch}{3em} % prevent overfull lines
\providecommand{\tightlist}{%
  \setlength{\itemsep}{0pt}\setlength{\parskip}{0pt}}
\setcounter{secnumdepth}{-\maxdimen} % remove section numbering
\usepackage{booktabs}
\usepackage{longtable}
\usepackage{array}
\usepackage{multirow}
\usepackage{wrapfig}
\usepackage{float}
\usepackage{colortbl}
\usepackage{pdflscape}
\usepackage{tabu}
\usepackage{threeparttable}
\usepackage{threeparttablex}
\usepackage[normalem]{ulem}
\usepackage{makecell}
\usepackage{xcolor}
\ifLuaTeX
  \usepackage{selnolig}  % disable illegal ligatures
\fi

\title{Intro to Urban Analytics}
\author{Bon Woo Koo \& Subhrajit Guhathakurta}
\date{2022-08-20}

\begin{document}
\maketitle

This course introduces students to the field of urban analytics. The
main objective of this course is for students to master important
theories and concepts emerging in the field of urban analytics. Students
will complete this course with a working knowledge of how data and
advanced analytical techniques can enhance the planning and operation of
cities.

\hypertarget{prerequisites}{%
\section{Prerequisites}\label{prerequisites}}

\begin{itemize}
\tightlist
\item
  Basic understanding of geographic information systems (GIS) and
  applied statistics
\item
  Working knowledge of any programming language, preferably the R (or
  Python)
\end{itemize}

\hypertarget{grading-breakdown}{%
\section{Grading breakdown}\label{grading-breakdown}}

There are four major assignments, four mini assignments, and one final
team project. Only three out of the four major assignments will be
counted towards the final grade. Same applies to the mini assignments.

\begin{table}
\centering
\begin{tabular}[t]{l|l}
\hline
Assignment.Type & Percent\\
\hline
Final project & 25\% (up to X number of people)\\
\hline
Major assignment & 45\% (15\% each x 3)\\
\hline
Mini assignment & 30\% (10\% each x 3)\\
\hline
\end{tabular}
\end{table}

The final grade will be assigned as a letter grade according to the
following scale:

\begin{itemize}
\tightlist
\item
  \textbf{A \(~~~\) 100\%-90\%} \(~~~\) Excellent (4 quality points per
  credit hour)
\item
  \textbf{B \(~~~\) 89\% - 80\%} \(~~~\) Good (3 quality points per
  credit hour)
\item
  \textbf{C \(~~~\) 79\% - 70\%} \(~~~\) Satisfactory (2 quality points
  per credit hour)
\item
  \textbf{D \(~~~\) 69\% - 60\%} \(~~~\) Passing (1 quality points per
  credit hour)
\item
  \textbf{F \(~~~\) 59\% \(~\)-\(~\) 0\%} \(~~~\) Failure (0 quality
  points per credit hour)
\end{itemize}

\hypertarget{course-schedules}{%
\section{Course schedules}\label{course-schedules}}

\begin{verbatim}
## Warning in (function (..., deparse.level = 1) : number of columns of result is
## not a multiple of vector length (arg 1)
\end{verbatim}

\begin{table}
\centering
\begin{tabular}[t]{>{\raggedright\arraybackslash}p{12em}|l|l|l|l}
\hline
Module & Week & Topic & Reading & Assignment\\
\hline
 & 1 & Intro to Urban Analytics in R (Slide), <br> Intro to R - 1 (RMD) & [1](https://journals.sagepub.com/doi/full/10.1177/2399808319839494) & [Survey](https://forms.gle/4BBtMfrvr5yJLebK9) & Preparation\\
\cline{2-5}
\multirow[t]{-2}{12em}{\raggedright\arraybackslash Preparation} & 2 & Modern urban data vs. conventional data (Slide), <br> Intro to R - 2 (RMD) &  &  & Preparation\\
\cline{1-5}
 &  & 3 & Accessing data (Slide), Census API (RMD), <br> Yelp API (RMD) &  & Mini\\
\cline{3-5}
 &  & 4 & Tidy data (Slide), <br> Data wrangling (RMD) & [1](https://r4ds.had.co.nz/tidy-data.html), [2](https://r4ds.had.co.nz/transform.html) & Mini\\
\cline{3-5}
 &  & 5 & First statistical insights from your data (Slide), <br> Hands-on (RMD) &  & Mini\\
\cline{3-5}
\multirow[t]{-4}{12em}{\raggedright\arraybackslash Module 1: <br> A walk-through of your first UA project - <br> POI} & \multirow[t]{-4}{*}{\raggedright\arraybackslash Census} & 6 & Interactive visualization (Slide), <br> Hands-on (RMD) &  & Mini\\
\cline{1-5}
 & 7 & General Transit Feed Specification (Slide), <br> [GTFS and equity (RMD)](./Lab/module_2/1_GTFS.html) &  &  & Module 2: <br> Transportation\\
\cline{2-5}
\multirow[t]{-2}{12em}{\raggedright\arraybackslash Module 2: <br> Transportation} & 8 & Open Street Map (Slide), <br> [OSM as a graph (RMD)](./Lab/module_2/2_OSM.html) &  & Major & Module 2: <br> Transportation\\
\cline{1-5}
 &  & 9 & Urban images and computer vision (Slide), <br> Yolo in R (RMD) &  & \\
\cline{3-5}
\multirow[t]{-2}{12em}{\raggedright\arraybackslash Module 3: <br> Image} & \multirow[t]{-2}{*}{\raggedright\arraybackslash computer vision} & 10 & Many ways of viewing your city (Slide), <br> Sampling images and using Vision API (RMD) &  & Major\\
\cline{1-5}
 & 11 & Twitter API (Slide), &  &  & Module 4: <br> Social media\\
\cline{2-5}
\multirow[t]{-2}{12em}{\raggedright\arraybackslash Module 4: <br> Social media} & 12 & Preprocessing texts (Slide), <br> Sentiment analysis (RMD) &  & Major & Module 4: <br> Social media\\
\cline{1-5}
 & 13 & Storytelling with data - 1 &  &  & Module 5: <br> Storytelling\\
\cline{2-5}
\multirow[t]{-2}{12em}{\raggedright\arraybackslash Module 5: <br> Storytelling} & 14 & Storytelling with data - 2 &  & Major & Module 5: <br> Storytelling\\
\cline{1-5}
Student Presentations & 15 & Student presentations &  &  & Student Presentations\\
\cline{1-5}
 & 16 & Student presentations - continued &  &  & Reading weeks\\
\cline{2-5}
\multirow[t]{-2}{12em}{\raggedright\arraybackslash Reading weeks} & 17 & Wrap up &  &  & Reading weeks\\
\hline
\end{tabular}
\end{table}

\hypertarget{how-to-succeed-in-this-class}{%
\section{How to succeed in this
class}\label{how-to-succeed-in-this-class}}

\begin{enumerate}
\def\labelenumi{\arabic{enumi}.}
\tightlist
\item
  Be prepared for occasional frustration. It's part of learning process.
  However, don't spin the wheel. You are responsible for actively
  searching for help. Don't wait until the last minute (e.g., homework).
\item
  Read assigned book chapters/materials, review their examples and
  snippets, replicate their results, and repeat until you understand.
\item
  Work with peers. Form a group early in the semester, and have their
  sharp eyes on your code. Still, you need to submit your HW
  individually.
\item
  If you have a trouble with your code outside of class (and get
  frustrated), Google it. It will not only be faster and more efficient
  than contacting us, but trouble-shooting on your own is essential
  skill, particularly after you graduate. Luckily, most of the problems
  you may encounter in this class have been already encountered by
  others. You can search how they solved them in \textbf{StackOverFlow.}
\item
  Of course, you can ask questions to us anytime, inside or outside
  classroom. I strongly encourage you to utilize our office hours as
  another learning opportunity.
\end{enumerate}

\hypertarget{textbooksresources}{%
\section{Textbooks/resources}\label{textbooksresources}}

There is no textbook associated with this course. I highly recommend
Data Action by Sarah Williams, and Urban Analytics by Alex Singleton,
Seth Spielman and David Folch is another popular textbook on the topic.

\begin{itemize}
\tightlist
\item
  \href{https://r4ds.had.co.nz/}{R for Data Science}: free online
\item
  \href{https://rpubs.com/spring19cp6521/Syllabus}{Geocomputation with
  R}: free online
\item
  \href{https://github.com/alexsingleton/urban_analytics}{Urban
  Analytics - supporting materials}
\end{itemize}

\hypertarget{technology}{%
\section{Technology}\label{technology}}

Cell phone use is prohibited at all times during class, except if you
are using cell phones to answer quizzes/ surveys. Laptops, tablets,
e-readers, and other digital devices may be used to take notes or refer
to relevant information, take quizzes, and complete in-class
assignments. If you are using a digital device for non-course purposes
at any time during the semester, you will be asked to refrain from using
it for the remainder of the course. No exceptions.

There will be times in class when the instructor reserves the right to
enact the ``No Device Rule.'' During these times, all digital devices
will be required to be stored off desks so that students may concentrate
on tasks or presentations. Expect that this rule will be used when your
peers are presenting and during guest lectures.

\hypertarget{student-faculty-expectations}{%
\section{Student-Faculty
expectations}\label{student-faculty-expectations}}

At Georgia Tech, we believe that it is important to continually strive
for an atmosphere of mutual respect, acknowledgement, and responsibility
between faculty members and the student body. See
\url{http://www.catalog.gatech.edu/rules/22.php} for an articulation of
some basic expectations---that you can have of me, and that I have of
you. Respect for knowledge, hard work, and cordial interactions will
help build the environment we seek. Therefore, I encourage you to remain
committed to the ideals of Georgia Tech while in this class.

\hypertarget{academic-integrity}{%
\section{Academic integrity}\label{academic-integrity}}

Georgia Tech aims to cultivate a community based on trust, academic
integrity, and honor. Students are expected to act according to the
highest ethical standards. For more information on Georgia Tech's
Academic Honor Code, please visit
\url{http://www.catalog.gatech.edu/rules/18b.php} and
\url{http://www.catalog.gatech.edu/genregulations/honorcode.php}.

\hypertarget{ada-accommodations}{%
\section{ADA accommodations}\label{ada-accommodations}}

If you are a student with learning needs that require special
accommodation, contact the Office of Disability Services at
(404)894-2563 or \url{http://disabilityservices.gatech.edu/}, as soon as
possible, to make an appointment to discuss your special needs and to
obtain an accommodations letter. Please also e-mail me as soon as
possible in order to set up a time to discuss your learning needs.

\end{document}
